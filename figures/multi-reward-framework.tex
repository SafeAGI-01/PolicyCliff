\begin{figure}[h!]
  \centering
  \begin{tikzpicture}[
    font=\sffamily,
    node distance=0.7cm and 1.5cm,
    >=Latex,
    every node/.style={align=center, rounded corners=3pt, thick},
    data/.style={draw, fill=gray!15, minimum width=2cm, minimum height=1cm, font=\small},
    policy/.style={draw, fill=blue!15, minimum width=2.5cm, minimum height=6cm, font=\normalsize},
    reward/.style={draw, fill=orange!15, minimum width=2cm, minimum height=1cm, font=\small},
    obj/.style={draw, fill=red!15, minimum width=2cm, minimum height=1cm, font=\small}
  ]

    % 中心LLM节点 (居中调整)
    \node[policy] (LLM) at (0,-0.75) {\textbf{LLM}};

    % 左侧数据节点
    \node[data,left=of LLM,yshift=2.5cm] (D1) {$\mathcal{D}_1$};
    \node[data,left=of LLM] (D2) {$\mathcal{D}_2$};
    \node[below=0.35cm of D2,font=\normalsize] (dotsD) {$\vdots$};
    \node[data,below=0.35cm of dotsD] (DN) {$\mathcal{D}_N$};

    % 右侧奖励节点
    \node[reward,right=of LLM,yshift=2.5cm] (R1) {$\RewardFunc_1$};
    \node[reward,right=of LLM] (R2) {$\RewardFunc_2$};
    \node[below=0.35cm of R2,font=\normalsize] (dotsR) {$\vdots$};
    \node[reward,below=0.35cm of dotsR] (RN) {$\RewardFunc_N$};

    % 目标节点
    \node[obj,right=1.5cm of R2] (J) {$J(\pi)$};

    % % 箭头:数据到LLM
    % \foreach \d in {D1,D2,DN}{
    %   \draw[->,thick] (\d.east)--(LLM.west);
    % }
    % \node[font=\scriptsize,above] at ($(D2.east)!0.5!(LLM.west)$) {prompts};

    % % 箭头:LLM到奖励(沿箭头方向标注)
    % \foreach \r in {R1,R2,RN}{
    %   \draw[->,thick] (LLM.east)--(\r.west) node[midway,above,sloped,font=\scriptsize] {rollout};
    % }

     % 箭头:数据到LLM(平行三条线)
    \draw[->,thick] (D1.east) -- ([yshift=2.5cm]LLM.west)
    node[midway,above,sloped,font=\small] {prompts};
    \draw[->,thick] (D2.east) -- (LLM.west)
    node[midway,above,sloped,font=\small] {prompts};
    \draw[->,thick] (DN.east) -- ([yshift=-2.5cm]LLM.west)
    node[midway,above,sloped,font=\small] {prompts};

    % 箭头:LLM到奖励(平行三条线)
    \draw[->,thick] ([yshift=2.5cm]LLM.east) -- (R1.west) node[midway,above,sloped,font=\small] {rollout};
    \draw[->,thick] (LLM.east) -- (R2.west) node[midway,above,sloped,font=\small] {rollout};
    \draw[->,thick] ([yshift=-2.5cm]LLM.east) -- (RN.west) node[midway,above,sloped,font=\small] {rollout};

    % 箭头:奖励到目标
    \foreach \r in {R1,R2,RN}{
      \draw[->,thick] (\r.east)--(J.west);
    }
    \node[font=\small,above] at ($(R2.east)!0.5!(J.west)$) {aggregate};

    % 调整反馈循环箭头位置 (避免重叠)
    \draw[->,thick] (J.south) |- ++(0,-3.0cm) -| (LLM.south) node[below=0.45cm of RN,font=\small]{backpropagation \& update policy};

  \end{tikzpicture}
  \vspace{0.2cm}
  \caption{The framework for training a single LLM policy with multiple data classes and specialized reward models. Prompts from data classes $\mathcal{D}_k$ are used to generate trajectories (rollouts) with the policy $\pi$. Each trajectory is evaluated by its corresponding reward model $\RewardFunc_k$. The rewards are aggregated into a global objective function $J(\pi)$, which is then used to update the policy via backpropagation.}
  \label{fig:multi-reward-framework}
\end{figure}